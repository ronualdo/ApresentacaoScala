\begin{frame}{Principais conceitos}
	\begin{block}{Funções como entidades de primeira ordem} 
		\begin{itemize}
			\item As funções podem ser criadas em qualquer lugar do programa
			\item As funções podem armazenadas em estruturas de dados
			\item As funções podem ser passadas como argumento para outras funções
			\item As funções podem retornar outras funções como resultados
		\end{itemize}
	\end{block}
	
	\begin{center}
		\lstinputlisting{conceitos/exemplos/funcoes.scala}
	\end{center}
\end{frame}

\begin{frame}{Principais conceitos}
	\begin{block}{Funções de alta-ordem}
		\begin{itemize}
			\item Recebem uma ou mais funções como entrada
			\item Retornam uma função como saída
			\begin{itemize}
				\item Exemplo: map
			\end{itemize}
		\end{itemize}
	\end{block}
\end{frame}

%Uma closure ocorre normalmente quando uma função é declarada dentro do corpo de outra,
%e a função interior referencia variáveis locais da função exterior. Em tempo de execução, 
% quando a função exterior é executada, então uma closure é formada, que consiste do código 
%da função interior e referências para quaisquer variáveis no escopo da função exterior que a closure necessita.
\begin{frame}{Principais conceitos}
	\begin{block}{Closures}
		\begin{itemize}
			\item Um closure é uma função que referencia variáveis livres no contexto léxico. 
			\item Função anônima definida on-the-fly dentro de uma função ou expressão
			\item Uma closure pode ser definida pelo compilador como resultado de outra função
			\item Closure é diferente de uma função anônima
		\end{itemize}
	\end{block}
	
	\begin{center}
		\lstinputlisting{conceitos/exemplos/closures.scala}
	\end{center}
\end{frame}


\begin{frame}{Principais conceitos}
	\begin{block}{Currying}
		%technique of transforming a function that takes multiple arguments 
		% (or an n-tuple of arguments) in such a way that it can be called as 
		% a chain of functions each with a single argument.
% 		Uma função "curried" transforma uma função que recebe múltiplos parâmetros em uma função que 
		%pode ser chamada como uma cadeia de funções que recebem somente um parâmetro
		%cada vez.
		Currying é o processo de transformar uma função que recebe múltiplos
		argumentos em uma função que recebe um único argumento e retorna outra função.
	\end{block}
	%\begin{block}{Visão informal}
	%	Uma função "curried" quando ela possui mais de uma lista de parâmetros, sendo
	%	assim, a mesma função
	%	é invocada em cadeia para cada lista de parâmetro.
	%\end{block}
	
	\begin{center}
		\lstinputlisting{conceitos/exemplos/curry.scala}
	\end{center}
\end{frame}

\begin{frame}{Tuplas}
	\begin{block}{Tuplas}
		Tupla é tipo de coleção imutavel, que pode conter diferentes tipos de
		elementos.
	\end{block}
	
	\begin{center}
		\lstinputlisting{conceitos/exemplos/tupla.scala} 
	\end{center}
\end{frame}
 
\begin{frame}{Principais conceitos}
	\begin{block}{Pattern Matching}
	Pattern matching é uma forma de associar nomes a coisas e possivelmente quebrar as expressões
	em sub-expressões ao mesmo tempo em que associa cada sub-expressão a cosias.
		\begin{itemize}
			\item Reconhecer valores
			\item Associar variáveis para reconhecer valores
			\item ``Quebrar" valores em partes
		\end{itemize}
	\end{block}
	
	\begin{center}
		\lstinputlisting{conceitos/exemplos/match.scala}
	\end{center}
\end{frame}